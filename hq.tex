\documentclass[10pt,a4paper]{article}
\usepackage[utf8]{inputenc}
\usepackage{amsmath}
\usepackage{amsfonts}
\usepackage{amssymb}
\usepackage{graphicx}
\title{Nirdesh 2024: HackerRank Questions}
\date{}
\begin{document}
\maketitle
\begin{itemize}
\item What’s Cooking Sunil?? \\ \\
Sunil is working in Canteen. His boss told him to serve the foods properly to professors and students. Sunil’s tray can hold a certain number of foods at a time. Sunil checks for the order, if the order is present in his tray. He serves it immediately and gets 5 stars. If it’s not there, he have to prepare and serve it, and for that he gets 3 stars. But the problem is that, each order has a different priority. If his tray is full, he has to serve that upcoming order keeping aside the least one.
Find how many stars Sunil would get at the end of the day. \\
Input format: \\
The first line contains an integer T, the no. of test cases. \\
The second line contains an integer N, the no. of orders Sunil gets. \\
The third line contains an integer K, the no. of orders can hold in his tray. \\
The fourth line contains space separated characters $a_1, a_2, a_3, ... a_n$ where $a_i$ are the orders labelled as characters. \\
The fifth line contains space separated characters $b_1, b_2, b_3 ... b_n$ where $b_i$ are the priorities of the orders of $a_1, a_2, a_3 ... a_n$  respectively labelled as characters. \\
Output format: \\
A single integer containing the stars Sunil would get. \\
Sample Input 0:
\begin{verbatim}
1 
8
4
a c a b d a e b
1 2 1 3 4 1 5 2
\end{verbatim}
Sample Output 0:
\begin{verbatim}
30
\end{verbatim}
\item Mirchi's Interview Round\\
Mohar got a interview call from Radio Mirchi. She comes for the interview at morning. But Mirchi team has something else in their mind.
They came to know that Mohar is very weak at maths. So, they want to chance of that. Now they want to test Mohar in a different way. They gave Mohar two numbers, X and Y. Mohar has been told that she has choice token, T, whose value is initially 0. She starts to count how many ways T numbers can be chosen from X numbers. Each time she finds the number of ways, the value of token increases by 1. The token has a power that the no. of value the token holds, Y will get multiplied with the count, the no. of times the value token holds. The value of token will increase up to X.\\
Now the Mirchi team wants know the sum of all possible counts Mohar would get.\\
Help Mohar to crack the interview.\\
Input Format:\\
The first line contains an integer N, the no. of test cases.\\
The second line two space separated integers, X and Y.\\
Output Format:\\
An integer.\\
Sample Input 0:
\begin{verbatim}
1
5 3
\end{verbatim}
Sample Output 0:
\begin{verbatim}
1024
\end{verbatim}
Explanation for Sample Output 0:\\
The possible counts Mohar would get are 1, 15, 90, 270, 405, 243.

\end{itemize}
\end{document}